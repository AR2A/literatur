%%%%%%%%%%%%%%%%%%%%%%%%%%%%%%%%%%%%%%%%%
% Beamer Presentation
% LaTeX Template
% Version 1.0 (10/11/12)
%
% This template has been downloaded from:
% http://www.LaTeXTemplates.com
%
% License:
% CC BY-NC-SA 3.0 (http://creativecommons.org/licenses/by-nc-sa/3.0/)
%
%%%%%%%%%%%%%%%%%%%%%%%%%%%%%%%%%%%%%%%%%

%----------------------------------------------------------------------------------------
%	PACKAGES AND THEMES
%----------------------------------------------------------------------------------------
\documentclass{beamer}

\mode<presentation> {

% The Beamer class comes with a number of default slide themes
% which change the colors and layouts of slides. Below this is a list
% of all the themes, uncomment each in turn to see what they look like.

%\usetheme{default}
%\usetheme{AnnArbor}
%\usetheme{Antibes}
%\usetheme{Bergen}
%\usetheme{Berkeley}
%\usetheme{Berlin}
%\usetheme{Boadilla}
%\usetheme{CambridgeUS}
%\usetheme{Copenhagen}
%\usetheme{Darmstadt}
%\usetheme{Dresden}
%\usetheme{Frankfurt}
%\usetheme{Goettingen}
%\usetheme{Hannover}
%\usetheme{Ilmenau}
%\usetheme{JuanLesPins}
%\usetheme{Luebeck}
\usetheme{Madrid}
%\usetheme{Malmoe}
%\usetheme{Marburg}
%\usetheme{Montpellier}
%\usetheme{PaloAlto}
%\usetheme{Pittsburgh}
%\usetheme{Rochester}
%\usetheme{Singapore}
%\usetheme{Szeged}
%\usetheme{Warsaw}

% As well as themes, the Beamer class has a number of color themes
% for any slide theme. Uncomment each of these in turn to see how it
% changes the colors of your current slide theme.

%\usecolortheme{albatross}
%\usecolortheme{beaver}
%\usecolortheme{beetle}
%\usecolortheme{crane}
%\usecolortheme{dolphin}
%\usecolortheme{dove}
%\usecolortheme{fly}
%\usecolortheme{lily}
%\usecolortheme{orchid}
%\usecolortheme{rose}
%\usecolortheme{seagull}
%\usecolortheme{seahorse}
%\usecolortheme{whale}
%\usecolortheme{wolverine}

%\setbeamertemplate{footline} % To remove the footer line in all slides uncomment this line
%\setbeamertemplate{footline}[page number] % To replace the footer line in all slides with a simple slide count uncomment this line

%\setbeamertemplate{navigation symbols}{} % To remove the navigation symbols from the bottom of all slides uncomment this line
}
%----------------------------------------------------------------------------------------
\usepackage{graphicx} % Allows including images
\usepackage{booktabs} % Allows the use of \toprule, \midrule and \bottomrule in tables
\setbeamerfont{caption}{size=\scriptsize}
\usepackage{hyperref}
\usepackage{listings}
%----------------------------------------------------------------------------------------
%	TITLE PAGE
%----------------------------------------------------------------------------------------
\title[]{ROS Basics - Topics, Services \& Messages} % The short title appears at the bottom of every slide, the full title is only on the title page
%----------------------------------------------------------------------------------------
\author{ARRA / AR2A} % Your name
\institute % Your institution as it will appear on the bottom of every slide, may be shorthand to save space
{
Advancements for Robotics in Rescue Applications
}
\date{\today} % Date, can be changed to a custom date
%----------------------------------------------------------------------------------------
\AtBeginSection{\frame{\sectionpage}}
%----------------------------------------------------------------------------------------
\begin{document}
%----------------------------------------------------------------------------------------
\begin{frame}
\titlepage % Print the title page as the first slide
\end{frame}
%----------------------------------------------------------------------------------------
%	PRESENTATION SLIDES
%----------------------------------------------------------------------------------------
\begin{frame}{Overview}
	ROS consists of 3 Levels:
	\begin{itemize}
		\item Filesystem
			\begin{itemize}
				\item Packages
				\item Disk-Files
				\item Message types
				\item ...
			\end{itemize}
		\item \underline{Computation Graph}
			\begin{itemize}
				\item Nodes
				\item Topcis
				\item Services
				\item ...
			\end{itemize}
		\item Community
			\begin{itemize}
				\item Repositories
				\item Wiki
				\item ...
			\end{itemize}
	\end{itemize}
\end{frame}

%----------------------------------------------------------------------------------------
\begin{frame}{Node}
	\begin{itemize}
		\item Process, which performs computation
		\item Control different parts of the robot (motors, rangefinder)
		\item Communication via messages/services
		\item Peer-to-peer network, master only registers nodes, topics, ...
		\item Identified by the graph resource name (hierarchical naming structure for topics/services/nodes/...)
		\item Name can be remapped at runtime
		\item Information about the running nodes can be shown with the tool ``rosnode``
	\end{itemize}
\end{frame}
%----------------------------------------------------------------------------------------
\begin{frame}{Message}
	\begin{itemize}
		\item Data structure, which can also be nested
		\item Standard data-types: int, float, string, time, array
		\item Special type ``Header``: contains timestamp and coordinate frame
		\item Simple textfiles: each line contains the type and name of a field
		\item Only useful for unidirectional communication; subscribers are unaware of all the publishers
		\item Information about messages can be shown with the tool ``rosmsg``
	\end{itemize}
\end{frame}
%----------------------------------------------------------------------------------------
\begin{frame}{Create message (1)}
	\begin{itemize}
		\item Create a .msg-file. For example:
		\lstinputlisting[frame=single, basicstyle=\footnotesize\ttfamily, language=C]{msgFile.txt}
		\item Naming convention for msg-types: name of the package + '/' + name of the .msg file\\
		\item \textbf{Example:} the file ``std\_msgs/msg/String.msg`` results in a message type of ``std\_msgs/String``
	\end{itemize}
\end{frame}

\begin{frame}{Create message (2)}
	\begin{itemize}
		\item Enable the generation of source code from the msg-files
		\item In the \textbf{package.xml} uncomment/add:
	\end{itemize}
	
	\lstinputlisting[frame=single, basicstyle=\footnotesize\ttfamily, language=C]{package.txt}

\end{frame}

\begin{frame}[fragile]
	\frametitle{Create message (3)}
	\begin{itemize}
		\item Example \textbf{CMakeList.txt} for a package with a message depending on std\_msgs
	\end{itemize}
	\lstinputlisting[frame=single, basicstyle=\footnotesize\ttfamily, language=C]{message.txt}		
\end{frame}

\begin{frame}{Topic}
	\begin{itemize}
		\item Simple string specified when publishing/subscribing to a message
	\end{itemize}
	\lstinputlisting[frame=single, basicstyle=\footnotesize\ttfamily, language=C]{topic.txt}
	\begin{itemize}
		\item The topic-type is defined by the message type $\rightarrow$ only one type per topic
		\item A subscriber will not accept a message of a different type
	\end{itemize}	
\end{frame}	

\begin{frame}{Publish to a topic}
	\begin{itemize}
		\item Example for publishing a standard message (string) to a topic
		\lstinputlisting[frame=single, basicstyle=\footnotesize\ttfamily, language=C]{topicPublish.txt}
		\item Advertise:
		\begin{itemize}
			\item The first parameter is the topic-name
			\item The second parameter specifies the maximum number of outgoing messages queued for delivery			
		\end{itemize}
	\end{itemize}
\end{frame}	

\begin{frame}{Subscribe to a topic}
	Function:
	\begin{itemize}
		\item ros::Subscriber subscribe( ... )
	\end{itemize}
	Parameters:
	\begin{itemize}
		\item \textbf{const std::string\& topic}: topic name
		\item \textbf{uint32\_t queue\_size}: queue for incoming messages (0 is infinite, but dangerous)
		\item \textbf{$<$callback$>$}: e.g. class method pointer and class instance pointer
		\item \textbf{const ros::TransportHints\& transport\_hints}: hint for the transport layer (preferred UDP transport, ...)
	\end{itemize}
\end{frame}

\begin{frame}{Service}
	\begin{itemize}
		\item Node offers service under a certain string name
		\item Provide service with ``ros::NodeHandle::advertiseService(string, callback)`` $\rightarrow$ provide a service name and a callback-function
		\item A client can make a persistent connection to a service, which enhances the performance
		\item ``rosservice``: tool for seeing a list of available services and making queries
		\item ``rossrv``: tool, which displays information about .srv data structures
	\end{itemize}
\end{frame}	

\begin{frame}{Create service (1)}
	\begin{itemize}
		\item Create .srv-file
		\item Structure for request and reply (similar to a remote procedure call)
		\lstinputlisting[frame=single, basicstyle=\footnotesize\ttfamily, language=C]{serviceCreate.txt}
		\item Service type similar to a message: package name + '/' + the name of the .srv file\\
		\textbf{Example:} ``my\_srvs/srv/MyService.srv`` results in the service type ``my\_srvs/MyService``
	\end{itemize}
\end{frame}

\begin{frame}{Create service (2)}
	Same process as with messages\\
	In the CMakeLists.txt:\\
	\begin{itemize}
		\item Additionally add the created .svg-files to the add\_message\_files() call\\
		to declare the service files to be built
	\end{itemize}
\end{frame}

\begin{frame}{Service: Generated Structure}
	\begin{itemize}
		\item Roscpp generates a structure with C++-Classes for a service:
		\lstinputlisting[frame=single, basicstyle=\footnotesize\ttfamily, language=C]{service.txt}	
	\end{itemize}
\end{frame}

\begin{frame}{Service: Calling a service}
	\begin{itemize}
		\item Call-method from the ``ros::service`` namespace\\
		``foo`` is the request/response structure
		\lstinputlisting[frame=single, basicstyle=\footnotesize\ttfamily, language=C]{callService.txt}	
		\item With node handle\\
		Enable a persistent connection with second parameter
		\lstinputlisting[frame=single, basicstyle=\footnotesize\ttfamily, language=C]{serviceCall.txt}
	\end{itemize}
\end{frame}
%\begin{frame}[allowframebreaks]{References}
%\scriptsize{\bibliographystyle{ieeetr}}
%\bibliography{references} %bibtex file name without .bib extension
%\end{frame}
%----------------------------------------------------------------------------------------
\begin{frame}
\Huge{\centerline{The End}}
\end{frame}
%----------------------------------------------------------------------------------------
\end{document} 
%----------------------------------------------------------------------------------------