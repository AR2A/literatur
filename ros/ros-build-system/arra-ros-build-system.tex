%%%%%%%%%%%%%%%%%%%%%%%%%%%%%%%%%%%%%%%%%
% Beamer Presentation
% LaTeX Template
% Version 1.0 (10/11/12)
%
% This template has been downloaded from:
% http://www.LaTeXTemplates.com
%
% License:
% CC BY-NC-SA 3.0 (http://creativecommons.org/licenses/by-nc-sa/3.0/)
%
%%%%%%%%%%%%%%%%%%%%%%%%%%%%%%%%%%%%%%%%%

%----------------------------------------------------------------------------------------
%	PACKAGES AND THEMES
%----------------------------------------------------------------------------------------
\documentclass{beamer}

\mode<presentation> {

% The Beamer class comes with a number of default slide themes
% which change the colors and layouts of slides. Below this is a list
% of all the themes, uncomment each in turn to see what they look like.

%\usetheme{default}
%\usetheme{AnnArbor}
%\usetheme{Antibes}
%\usetheme{Bergen}
%\usetheme{Berkeley}
%\usetheme{Berlin}
%\usetheme{Boadilla}
%\usetheme{CambridgeUS}
%\usetheme{Copenhagen}
%\usetheme{Darmstadt}
%\usetheme{Dresden}
%\usetheme{Frankfurt}
%\usetheme{Goettingen}
%\usetheme{Hannover}
%\usetheme{Ilmenau}
%\usetheme{JuanLesPins}
%\usetheme{Luebeck}
\usetheme{Madrid}
%\usetheme{Malmoe}
%\usetheme{Marburg}
%\usetheme{Montpellier}
%\usetheme{PaloAlto}
%\usetheme{Pittsburgh}
%\usetheme{Rochester}
%\usetheme{Singapore}
%\usetheme{Szeged}
%\usetheme{Warsaw}

% As well as themes, the Beamer class has a number of color themes
% for any slide theme. Uncomment each of these in turn to see how it
% changes the colors of your current slide theme.

%\usecolortheme{albatross}
%\usecolortheme{beaver}
%\usecolortheme{beetle}
%\usecolortheme{crane}
%\usecolortheme{dolphin}
%\usecolortheme{dove}
%\usecolortheme{fly}
%\usecolortheme{lily}
%\usecolortheme{orchid}
%\usecolortheme{rose}
%\usecolortheme{seagull}
%\usecolortheme{seahorse}
%\usecolortheme{whale}
%\usecolortheme{wolverine}

%\setbeamertemplate{footline} % To remove the footer line in all slides uncomment this line
%\setbeamertemplate{footline}[page number] % To replace the footer line in all slides with a simple slide count uncomment this line

%\setbeamertemplate{navigation symbols}{} % To remove the navigation symbols from the bottom of all slides uncomment this line
}
%----------------------------------------------------------------------------------------
\usepackage{graphicx} % Allows including images
\usepackage{booktabs} % Allows the use of \toprule, \midrule and \bottomrule in tables
\setbeamerfont{caption}{size=\scriptsize}
\usepackage{hyperref}
\usepackage{listings}
%----------------------------------------------------------------------------------------
%	TITLE PAGE
%----------------------------------------------------------------------------------------
\title[]{ROS Build System} % The short title appears at the bottom of every slide, the full title is only on the title page
%----------------------------------------------------------------------------------------
\author{ARRA / AR2A} % Your name
\institute % Your institution as it will appear on the bottom of every slide, may be shorthand to save space
{
\textbf{A}dvancements for \textbf{R}obotics in \textbf{R}escue \textbf{A}pplications
}
\date{\today} % Date, can be changed to a custom date
%----------------------------------------------------------------------------------------
\AtBeginSection{\frame{\sectionpage}}
%----------------------------------------------------------------------------------------
\begin{document}
%----------------------------------------------------------------------------------------
\begin{frame}
\titlepage
\end{frame}
%----------------------------------------------------------------------------------------
\begin{frame}
\tableofcontents
\end{frame}
%----------------------------------------------------------------------------------------
\begin{frame}{ARRA/AR2A}
\begin{large}\textbf{What do we want?}\end{large}
\begin{itemize}
 \item ARRA / AR2A aims to improve the current state of technology of robotics in rescue applications.
\end{itemize}
\begin{large}\textbf{Who are we?}\end{large}
\begin{itemize}
 \item A volunteer non-profit organisation of robotic enthusiasts.
\end{itemize}
\begin{large}\textbf{How can you help?}\end{large}
\begin{itemize}
 \item Check us out at \url{https://github.com/ar2a}
\end{itemize}
 \vspace{1cm}
\begin{large}\textbf{License information}\end{large}
\begin{itemize}
 \item \textbf{CC-BY-SA 4.0} \url{https://creativecommons.org/licenses/by-sa/4.0/}
\end{itemize}
\end{frame}
%----------------------------------------------------------------------------------------
\begin{frame}{Introduction 1}
 \begin{itemize}
 \item Software in ROS is organized in packages
 \item It provides a way to easily reuse ROS-Software
 \end{itemize}
 \vspace{1cm}
 \textbf{A ROS package might contain:}
 \begin{itemize}
 \item ROS nodes
 \item  a ROS-independent library
 \item a dataset
 \item configuration files
 \item a third-party piece of software
 \item or other useful modules that can be grouped together logically
 \end{itemize}
\end{frame}
%----------------------------------------------------------------------------------------
\begin{frame}{Introduction 2}
\begin{large}
\textbf{Why does ROS have a custom build system?}
\end{large}
\vspace{1cm}
\\For development of single software projects, existing tools like Autotools, CMake, and the build systems included with IDEs tend to be sufficient.
\newline
\newline
ROS is a very large collection of loosely federated packages. That means lots of independent packages which depend on each other, utilize various programming languages, tools, and code organization conventions.
\end{frame}
%----------------------------------------------------------------------------------------
\section{rosbuild\cite{ROS:2015:Online}}
%----------------------------------------------------------------------------------------
\begin{frame}{Utilities}
\begin{large}\textbf{Tools for installing, building and locating of packages}\end{large}

\begin{itemize}
	\item Build: 							rosmake
	\item Install:							rosinstall
	\item Searching for packages:			roslocate
	\item Install thirdparty libraries:		rosdep
	\item Packages:							rospack, roscd
	\item Stacks:							rosstack, roscd
\end{itemize}
\end{frame}
%----------------------------------------------------------------------------------------
\begin{frame}{rosmake 1}
	
\begin{large}\textbf{Build: rosmake}\end{large}

\begin{itemize}
\item rosmake is a tool to assist with building ROS packages. It facilitates building packages that have dependencies.
\item ROS comprises a large number of packages. With the exception of some core packages that everything else depends on, many of the packages are largely independent. ROS provides build system to allow to only build what is actually necessary to run the packages.
\item A package may depend on any number of other packages, requiring that those packages be built first. These dependencies are specified in the package's manifest.xml file.
\item rosmake do the ROS-wide build of everything need for a given package.
\end{itemize} 
\end{frame}
%----------------------------------------------------------------------------------------
\begin{frame}{rosmake 2}
\begin{large}\textbf{Example: build move\_base}\end{large}
\lstinputlisting[frame=single, basicstyle=\footnotesize\ttfamily, language=C]{./listings/rosmake.txt}

rosmake will determine move\_base's dependencies (and the dependencies' dependencies), and will ensure that all dependencies are built prior to building move\_base.\\	
\begin{itemize}
	\item \url{http://wiki.ros.org/rosmake}
\end{itemize}
\end{frame}
%----------------------------------------------------------------------------------------
\begin{frame}
	\Huge{\centerline{R.I.P. rosbuild}}
\end{frame}	
%----------------------------------------------------------------------------------------
\section{catkin\cite{ROS:2015:Online}}
%----------------------------------------------------------------------------------------
\begin{frame}{Overview}
 \begin{large}\textbf{What is Catkin?}\end{large}
  \begin{definition}
   Catkin was designed to be more conventional than rosbuild, allowing for better distribution of packages, better cross-compiling support, and better portability.
  \end{definition}
\begin{itemize}
 \item Catkin is the official build system of ROS (which replaces rosbuild)
 \item It combines CMake macros and Python scripts to provide some functionality on top of CMake's normal workflow
 \item Support for automatic 'find package' infrastructure and building multiple, dependent projects at the same time
 \end{itemize}
\end{frame}
%----------------------------------------------------------------------------------------
\begin{frame}{General structure of a ROS Package}
\begin{figure}
  \includegraphics[width=0.6\textwidth]{./images/structure.pdf}
\end{figure}
% \begin{itemize}
 %\item include/package\_name: include C++ headers 
 %\item msg/: containing Message types
 %\item src/package\_name/:  Source files
 %\item srv/: containing Service types
 %\item scripts/: executable scripts
 %\item CMakeLists.txt:  CMake build file
 %\item package.xml 
 %\item CHANGELOG.rst
 %\end{itemize}
\end{frame}
%----------------------------------------------------------------------------------------
%\section{Concepts}
%----------------------------------------------------------------------------------------
\begin{frame}{Setup the built environment}
 \textbf{Creating a workspace}
 %\begin{itemize}
 %\item \$ source /opt/ros/indigo/setup.bash
 %\item \$ mkdir -p \textasciitilde/catkin\_ws/src
 %\item \$ cd \textasciitilde/catkin\_ws/src
 %\item \$ catkin\_init\_workspace
 %\end{itemize}
 \lstinputlisting[frame=single, basicstyle=\footnotesize\ttfamily, language=C]{./listings/setup_the_build_environment.txt}
 \textbf{Build empty project}
% \begin{itemize}
% \item \$ cd \textasciitilde/catkin\_ws/
% \item \$ catkin\_make
% \end{itemize}
  \lstinputlisting[frame=single, basicstyle=\footnotesize\ttfamily, language=C]{./listings/build_empty_project.txt}

 \textbf{Environment-Variables loaded?}
 %\begin{itemize}
 %\item \$ source devel/setup.bash
 %\item \$ echo \$ROS\_PACKAGE\_PATH
 %\item /home/youruser/catkin\_ws/src:/opt/ros/indigo/share: /opt/ros/indigo/stacks
 %\end{itemize}
  \lstinputlisting[frame=single, basicstyle=\footnotesize\ttfamily, language=C]{./listings/environment_variables_loaded.txt}

\end{frame}
%----------------------------------------------------------------------------------------
%----------------------------------------------------------------------------------------
\begin{frame}{Create and Build a ROS Package}
 \textbf{Create a new package called ’beginner\_tutorials’ which
depends on std\_msgs, roscpp, and rospy
}
 %\begin{itemize}
 %\item \$ catkin\_create\_pkg beginner\_tutorials std\_msgs rospy roscpp
 %\end{itemize}
   \lstinputlisting[frame=single, basicstyle=\footnotesize\ttfamily, language=C]{./listings/crate_new_package.txt}
 \vspace{5mm}
 \textbf{Build a package}
 %\begin{itemize}
 %\item \$ cd \textasciitilde/catkin\_ws/
 %\item \$ catkin\_make
 %\end{itemize}
    \lstinputlisting[frame=single, basicstyle=\footnotesize\ttfamily, language=C]{./listings/build_a_package.txt}
\end{frame}
%----------------------------------------------------------------------------------------
%----------------------------------------------------------------------------------------
\begin{frame}{Eclipse and ROS 1}
 \textbf{Creating the Eclipse project files}
  \lstinputlisting[frame=single, basicstyle=\footnotesize\ttfamily, language=C]{./listings/creating_the_eclipse_project_files.txt}
 \vspace{5mm}
 \textbf{Importing the project into Eclipse}
 \begin{itemize}
 \item Start Eclipse, select File  --\textgreater  Import. Select  Existing projects into workspace, hit next, then browse for your package's directory (select root directory). Do NOT select Copy projects into workspace. Then finish.
 \end{itemize}
\end{frame}
%----------------------------------------------------------------------------------------
%----------------------------------------------------------------------------------------
\begin{frame}{Eclipse and ROS 2}

 \textbf{Importing the project into Eclipse}

\begin{figure}[htb]
    \centering
    \begin{minipage}{0.45\linewidth}
        \centering
        \includegraphics[scale=0.3]{./images/import.pdf}
    \end{minipage}
    %\hfill
    \begin{minipage}{0.45\linewidth}
        \centering
        \includegraphics[scale=0.35]{./images/select.pdf}
    \end{minipage}
        \caption{Start Eclipse, select File  --\textgreater  Import. Select Existing projects into workspace, hit next.}
\end{figure}

\end{frame}
%----------------------------------------------------------------------------------------
%----------------------------------------------------------------------------------------
\begin{frame}{Eclipse and ROS 3}

\begin{figure}
	\centering
  \includegraphics[width=0.50\textwidth]{./images/file.pdf}
	\caption{Browse for your package's directory (select root directory). Do NOT select Copy projects into workspace. Then finish.}
	\label{fig1}
\end{figure}

\end{frame}
%----------------------------------------------------------------------------------------
%----------------------------------------------------------------------------------------
\begin{frame}[allowframebreaks]{References}
\scriptsize{\bibliographystyle{ieeetr}}
\bibliography{references} %bibtex file name without .bib extension
\end{frame}
%----------------------------------------------------------------------------------------
\begin{frame}
\Huge{\centerline{The End}}
\end{frame}
%----------------------------------------------------------------------------------------
\end{document} 
%----------------------------------------------------------------------------------------
