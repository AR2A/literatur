%%%%%%%%%%%%%%%%%%%%%%%%%%%%%%%%%%%%%%%%%
% Beamer Presentation
% LaTeX Template
% Version 1.0 (10/11/12)
%
% This template has been downloaded from:
% http://www.LaTeXTemplates.com
%
% License:
% CC BY-NC-SA 3.0 (http://creativecommons.org/licenses/by-nc-sa/3.0/)
%
%%%%%%%%%%%%%%%%%%%%%%%%%%%%%%%%%%%%%%%%%

%----------------------------------------------------------------------------------------
%	PACKAGES AND THEMES
%----------------------------------------------------------------------------------------
\documentclass{beamer}

\mode<presentation> {

% The Beamer class comes with a number of default slide themes
% which change the colors and layouts of slides. Below this is a list
% of all the themes, uncomment each in turn to see what they look like.

%\usetheme{default}
%\usetheme{AnnArbor}
%\usetheme{Antibes}
%\usetheme{Bergen}
%\usetheme{Berkeley}
%\usetheme{Berlin}
%\usetheme{Boadilla}
%\usetheme{CambridgeUS}
%\usetheme{Copenhagen}
%\usetheme{Darmstadt}
%\usetheme{Dresden}
%\usetheme{Frankfurt}
%\usetheme{Goettingen}
%\usetheme{Hannover}
%\usetheme{Ilmenau}
%\usetheme{JuanLesPins}
%\usetheme{Luebeck}
\usetheme{Madrid}
%\usetheme{Malmoe}
%\usetheme{Marburg}
%\usetheme{Montpellier}
%\usetheme{PaloAlto}
%\usetheme{Pittsburgh}
%\usetheme{Rochester}
%\usetheme{Singapore}
%\usetheme{Szeged}
%\usetheme{Warsaw}

% As well as themes, the Beamer class has a number of color themes
% for any slide theme. Uncomment each of these in turn to see how it
% changes the colors of your current slide theme.

%\usecolortheme{albatross}
%\usecolortheme{beaver}
%\usecolortheme{beetle}
%\usecolortheme{crane}
%\usecolortheme{dolphin}
%\usecolortheme{dove}
%\usecolortheme{fly}
%\usecolortheme{lily}
%\usecolortheme{orchid}
%\usecolortheme{rose}
%\usecolortheme{seagull}
%\usecolortheme{seahorse}
%\usecolortheme{whale}
%\usecolortheme{wolverine}

%\setbeamertemplate{footline} % To remove the footer line in all slides uncomment this line
%\setbeamertemplate{footline}[page number] % To replace the footer line in all slides with a simple slide count uncomment this line

%\setbeamertemplate{navigation symbols}{} % To remove the navigation symbols from the bottom of all slides uncomment this line
}
%----------------------------------------------------------------------------------------
\usepackage{graphicx} % Allows including images
\usepackage{booktabs} % Allows the use of \toprule, \midrule and \bottomrule in tables
\usepackage{subfigure}
\setbeamerfont{caption}{size=\scriptsize}
\usepackage{hyperref}
\usepackage{listings}
%----------------------------------------------------------------------------------------
%	TITLE PAGE
%----------------------------------------------------------------------------------------
\title[]{ROS Tools: roslaunch} % The short title appears at the bottom of every slide, the full title is only on the title page
%----------------------------------------------------------------------------------------
\author{ARRA / AR2A} % Your name
\institute % Your institution as it will appear on the bottom of every slide, may be shorthand to save space
{
\textbf{A}dvancements for \textbf{R}obotics in \textbf{R}escue \textbf{A}pplications
}
\date{\today} % Date, can be changed to a custom date
%----------------------------------------------------------------------------------------
\AtBeginSection{\frame{\sectionpage}}
%----------------------------------------------------------------------------------------
\begin{document}
%----------------------------------------------------------------------------------------
\begin{frame}
\titlepage % Print the title page as the first slide
\end{frame}
%----------------------------------------------------------------------------------------
%	PRESENTATION SLIDES
%----------------------------------------------------------------------------------------
\begin{frame}{Overview 1}
\begin{large}
	\textbf{roslaunch} \newline \newline
\end{large}
roslaunch is a tool for easily launching multiple ROS nodes locally and remotely via SSH, as well as setting parameters on the Parameter Server. \newline \newline
It takes in one or more XML configuration files (with the .launch extension) that specify the parameters to set and nodes to launch.
\end{frame}
%----------------------------------------------------------------------------------------
\begin{frame}{Overview 2}
	\begin{large}
		\textbf{Advantages:} \newline
	\end{large}
\begin{itemize}
	\item Starting implicitly the ROS "core", that's a collection of nodes and programs that are pre-requisites of a ROS-based system.
	\newline
	\item The launch file syntax itself is stable, and every effort will be made to provide backwards compatibility with new features. 
	\newline
	\item Option to set the necessary parameters for the single packages. 
\end{itemize}
\end{frame}
%----------------------------------------------------------------------------------------
\begin{frame}{Launch Files 1}
\begin{itemize}
 \item Many ROS packages come with "launch files", which can be runned with:
 \vspace{10px}
 \lstinputlisting[frame=single, basicstyle=\footnotesize\ttfamily, language=C]{./listings/launch.txt}
 \vspace{10px}
 These launch files usually bring up a set of nodes for the package that provide some aggregate functionality. 
\end{itemize}
\begin{itemize}
 \item Launch files can be found under the following path:
 \vspace{10px}
 \lstinputlisting[frame=single, basicstyle=\footnotesize\ttfamily, language=C]{./listings/launch_4.txt} 
\end{itemize}
\end{frame}
%----------------------------------------------------------------------------------------
\begin{frame}{Launch Files 2 - Example 1}
\begin{large}
	\textbf{Launch-File from example learning\_tf} 
\end{large}
\lstinputlisting[frame=single, basicstyle=\footnotesize\ttfamily, language=C]{./listings/launch_2.txt}
\end{frame}
%----------------------------------------------------------------------------------------
\begin{frame}{Launch Files 3 - Example 2}
	\begin{large}
		\textbf{Launch-File from example chickenonaraft} 
	\end{large}
	\vspace{10px}
\lstinputlisting[frame=single, basicstyle=\footnotesize\ttfamily, language=C]{./listings/launch_3.txt}
\end{frame}
%----------------------------------------------------------------------------------------
\begin{frame}{Launch file calling launch files}
\begin{itemize}
\item Large applications on a robot typically involve several interconnected nodes, each of which have many parameters. For this you can create a separately package only for the startup launch file:
\vspace{10px}
\lstinputlisting[frame=single, basicstyle=\footnotesize\ttfamily, language=C]{./listings/launch_5.txt}
\vspace{10px}
For the startup package are no further dependencies necessary. 
\end{itemize}
\begin{itemize}
 \item In the separately package the launch file calls the other launch files:
 \lstinputlisting[frame=single, basicstyle=\footnotesize\ttfamily, language=C]{./listings/launch_1.txt}
\end{itemize}
\end{frame}
%----------------------------------------------------------------------------------------
\begin{frame}
\Huge{\centerline{The End}}
\end{frame}
%----------------------------------------------------------------------------------------
\end{document} 
%----------------------------------------------------------------------------------------